%# -*- coding:utf-8 -*-
%% start of file `template_en.tex'.
%% Copyright 2006-1008 Xavier Danaux (xdanaux@gmail.com).
%
% This work may be distributed and/or modified under the
% conditions of the LaTeX Project Public License version 1.3c,
% available at http://www.latex-project.org/lppl/.


\documentclass[11pt,a4paper]{moderncv}

\usepackage{fontspec,xunicode}
\setmainfont{Tahoma}
\usepackage[slantfont,boldfont]{xeCJK}
\usepackage{xcolor}                 % replace by the encoding you are using


\setmainfont{Times New Roman}%缺省英文字体.serif是有衬线字体sans serif无衬线字体
\setCJKmainfont[ItalicFont={Kai}, BoldFont={Hei}]{STSong}%衬线字体 缺省中文字体为
\setCJKsansfont{STSong}
\setCJKmonofont{STFangsong}%中文等宽字体
%-----------------------xeCJK下设置中文字体------------------------------%
\setCJKfamilyfont{song}{SimSun}                             %宋体 song
\newcommand{\song}{\CJKfamily{song}}
\setCJKfamilyfont{fs}{FangSong_GB2312}                      %仿宋2312 fs
\newcommand{\fs}{\CJKfamily{fs}}
\setCJKfamilyfont{yh}{Microsoft YaHei}                    %微软雅黑 yh
\newcommand{\yh}{\CJKfamily{yh}}
\setCJKfamilyfont{hei}{SimHei}                              %黑体  hei
\newcommand{\hei}{\CJKfamily{hei}}
\setCJKfamilyfont{hwxh}{STXihei}                                %华文细黑  hwxh
\newcommand{\hwxh}{\CJKfamily{hwxh}}
\setCJKfamilyfont{asong}{Adobe Song Std}                        %Adobe 宋体  asong
\newcommand{\asong}{\CJKfamily{asong}}
\setCJKfamilyfont{ahei}{Adobe Heiti Std}                            %Adobe 黑体  ahei
\newcommand{\ahei}{\CJKfamily{ahei}}
\setCJKfamilyfont{akai}{Adobe Kaiti Std}                            %Adobe 楷体  akai
\newcommand{\akai}{\CJKfamily{akai}}


%------------------------------设置字体大小------------------------%
\newcommand{\chuhao}{\fontsize{42pt}{\baselineskip}\selectfont}     %初号
\newcommand{\xiaochuhao}{\fontsize{36pt}{\baselineskip}\selectfont} %小初号
\newcommand{\yihao}{\fontsize{28pt}{\baselineskip}\selectfont}      %一号
\newcommand{\erhao}{\fontsize{21pt}{\baselineskip}\selectfont}      %二号
\newcommand{\xiaoerhao}{\fontsize{18pt}{\baselineskip}\selectfont}  %小二号
\newcommand{\sanhao}{\fontsize{15.75pt}{\baselineskip}\selectfont}  %三号
\newcommand{\sihao}{\fontsize{14pt}{\baselineskip}\selectfont}         %四号
\newcommand{\xiaosihao}{\fontsize{12pt}{\baselineskip}\selectfont}  %小四号
\newcommand{\wuhao}{\fontsize{10.5pt}{\baselineskip}\selectfont}    %五号
\newcommand{\subwuhao}{\fontsize{10pt}{\baselineskip}\selectfont}    %次五号
\newcommand{\xiaowuhao}{\fontsize{9pt}{\baselineskip}\selectfont}   %小五号
\newcommand{\liuhao}{\fontsize{7.875pt}{\baselineskip}\selectfont}  %六号
\newcommand{\qihao}{\fontsize{5.25pt}{\baselineskip}\selectfont}    %七号


%\usepackage{fontawesome}
% \setCJKmainfont[BoldFont={WenQuanYi Micro Hei/Bold}]{WenQuanYi Micro Hei}
%\defaultfontfeatures{Mapping=tex-text}
%\XeTeXlinebreaklocale "zh"
%\XeTeXlinebreakskip = 0pt plus 1pt minus 0.1pt
% moderncv themes
\moderncvtheme[blue]{classic}                 % optional argument are 'blue' (default), 'orange', 'red', 'green', 'grey' and 'roman' (for roman fonts, instead of sans serif fonts)
%\moderncvtheme[green]{classic}                % idem
%\moderncvtheme[blue,roman]{hht}
% character encoding



% adjust the page margins
\usepackage[scale=0.9]{geometry}
%\setlength{\hintscolumnwidth}{3cm}						% if you want to change the width of the column with the dates
%\AtBeginDocument{\setlength{\maketitlenamewidth}{6cm}}  % only for the classic theme, if you want to change the width of your name placeholder (to leave more space for your address details
\AtBeginDocument{\recomputelengths}                     % required when changes are made to page layout lengths

% personal data
\firstname{王辰翔}
\familyname{}
% \title{SanZhang}               % optional, remove the line if not wanted
\address{河北省廊坊市三河市燕郊经济开发区}{}    % optional, remove the line if not wanted
%\address{1990/11/11}{}    % optional, remove the line if not wanted
\mobile{17331624714}                    % optional, remove the line if not wanted
%\fax{fax (optional)}                          % optional, remove the line if not wanted
\email{ndyttl0903@gmail.com}                     % optional, remove the line if not wanted
%\homepage{Blog: http://geekplux.com} % optional, remove the line if not wanted
%\social[github]{GitHub: https://github.com/geekplux}
\extrainfo{%
  WeChat: Ndywcxpope \\
  QQ: 123456
}

\photo[64pt]{头像.png}                         % '64pt' is the height the picture must be resized to and 'picture' is the name of the picture file; optional, remove the line if not wanted
%\quote{China\TeX 您的LaTeX乐园,TeX\&\LaTeX 王国}                 % optional, remove the line if not wante

%\nopagenumbers{}                             % uncomment to suppress automatic page numbering for CVs longer than one page


%----------------------------------------------------------------------------------
%            content
%----------------------------------------------------------------------------------
\begin{document}
\maketitle
\vspace*{-14mm}

% \section{工作经历}
% \cventry{17.01-17.12}{不长不短的公司名称}{可爱的项目}{https://keaidexiangmu.com}{}{工作中,
%   我负责了 xxx 的开发和维护,运用 yyy 技术解决了 zzz 的重大问题。积极参与开源社
%   区贡献}
% \cventry{16.01-16.12}{好长的公司名称}{有趣的项
%   目}{http://www.youqudexiangmu.com}{}{工作中,我按照领导的要求编程,做出了让领
%   导满意的作品,为公司做出了贡献。我觉得我还是凑字数吧,编不下去了}
% \cventry{15.01-15.12}{不知道叫什么的公司名称}{不可告人的项
%   目}{http://bukegaoren.com}{}{独立编写了根本编不下去的项目简介,可能还是凑字数
%   比较好,来凑字数吧来凑字数吧来凑字数吧。出色的完成了凑字数的工作,并获得了最佳
%   凑字数员工奖}
\section{教育经历}
\cventry{11.09-15.06}{本科}{尼姑庵大学}{计算机科学与技术}{}{}                % arguments 3 to 6 are optional
\cvlistitem{最快编程大师一等奖}
\cvlistitem{最强编程大师金奖}
\cvlistitem{第 x 届「编程杯」gayhub 赛区一等奖}
\cvlistitem{国家奖学金/三好学生/学生会主席/\emph{获得女朋友一个}}
\cventry{15.09-18.06}{硕士}{和尚庙大学}{软件工程}{实验室 XXX 导师 XXX}{主要研究了
  人工智能,图形学,编译原理,机械键盘的拆装,快递包装的暴力拆解,颈椎与视觉保
  养,抹平小腹,治疗腰椎间盘突出}                % arguments 3 to 6 are optional
\section{项目}
\subsection{科研项目}
\cvline{FFF-TNT}{建立深度学习模型,通过训练单身狗愤怒指数,预测 TNT 威力值,达到
  最后 FFF 全部的效果}
\cvline{NASA}{我又编不下去了,还是凑个字数,要不然你们将来替换也麻烦}
\subsection{开源项目}
\cvline{markvis}{在 markdown 中直接生成可视化图表的插件
  \emph{https://markvis.js.org} \textbf{GitHub 1000 stars}}
\cvline{netjsongraph.js}{用力导向图可视化出无线路由图谱数据 \emph{https://github.com/netjson/netjsongraph.js}}
\cvline{typing}{Hexo 静态博客主题 \emph{https://github.com/geekplux/hexo-theme-typing}}
\cvline{UnityVis}{Unity 中的基本可视化图表 \emph{https://github.com/geekplux/Basic-Visualization-in-Unity}}
\section{技能}
\cvline{\textbf{前端}}{熟练掌握前端该会的东西,熟悉前端该熟悉的东西,了解前端该
  了解的原理,熟练使用前端该熟练使用的各类工具}
\cvline{\textbf{后端}}{熟练掌握后端该会的东西,熟悉后端该熟悉的东西,了解后端该了解的原理}
\cvline{\textbf{数据}}{掌握处理数据,掌握数据库}
\cvline{\textbf{其他}}{熟练使用 Git/Vim/Emacs/Makefile}

\section{Publications}
\cvline{已录用}{张三,李四,王麻子 基于 latex 的简历凑字数研究[C]// CVChina. 2017.}

% \subsection{Vocational}
% \cventry{year--year}{Job title}{Employer}{City}{}{Description}                % arguments 3 to 6 are optional
% \cventry{year--year}{Job title}{Employer}{City}{}{Description}                % arguments 3 to 6 are optional
% \subsection{Miscellaneous}
% \cventry{year--year}{Job title}{Employer}{City}{}{Description line 1\newline{}Description line 2}% arguments 3 to 6 are optional

% \section{Languages}
% \cvlanguage{language 1}{Skill level}{Comment}
% \cvlanguage{language 2}{Skill level}{Comment}
% \cvlanguage{language 3}{Skill level}{Comment}

% \section{Computer skills}
% \cvcomputer{category 1}{XXX, YYY, ZZZ}{category 4}{XXX, YYY, ZZZ}
% \cvcomputer{category 2}{XXX, YYY, ZZZ}{category 5}{XXX, YYY, ZZZ}
% \cvcomputer{category 3}{XXX, YYY, ZZZ}{category 6}{XXX, YYY, ZZZ}

% \section{Interests}
% \cvline{篮球}{\small 体力与技巧}
% \cvline{hobby 2}{\small Description}
% \cvline{hobby 3}{\small Description}

% \renewcommand{\listitemsymbol}{-} % change the symbol for lists

% \section{Extra 1}
% \cvlistitem{Item 1}
% \cvlistitem{Item 2}
%\cvlistitem[+]{Item 3}            % optional other symbol% XeLaTeX can use any Mac OS X font. See the setromanfont command below.
% Input to XeLaTeX is full Unicode, so Unicode characters can be typed directly into the source.

% The next lines tell TeXShop to typeset with xelatex, and to open and save the source with Unicode encoding.

%!TEX TS-program = xelatex
%!TEX encoding = UTF-8 Unicode

%\section{Extra 2}
%\cvlistdoubleitem[\Neutral]{Item 1}{Item 4}
%\cvlistdoubleitem[\Neutral]{Item 2}{Item 5}
%\cvlistdoubleitem[\Neutral]{Item 3}{}

%% Publications from a BibTeX file
%\nocite{*}
%\bibliographystyle{plain}
%\bibliography{publications}       % 'publications' is the name of a BibTeX file

% \begin{thebibliography}{}
% \bibitem[]{} 移动增强现实可视化综述[C]. ChinaVis 2017.
% \end{thebibliography}


\end{document}


%% end of file `template_en.tex'.

%%% Local Variables:
%%% mode: latex
%%% TeX-command-extra-options: "-shell-escape"
%%% TeX-master: t
%%% TeX-engine: xetex
%%% End: